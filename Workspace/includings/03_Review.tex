\section{Representation Learning for Time Series Data}\label{review}
In this chapter the found literature is put into context. Starting with classical literature about the fundamental findings followed by actual trends in the Area of Representation Learning. Finally the different Representation Learning Strategies are listed and compared.
\subsection{Historical view}
In this chapter the fundamental literature about the topic is going to be discussed. \\
Sensors and comparable applications produce time series data points which on a closer look may not make sense. They can vary in an unforeseen way and for a short time window they may be completely random. We have to step back and observe longer time periods which could be days or weeks. Or, for very dense measuring it is shorter but there are way more data points to handle.\\
Sometimes it is possible for a human to see some patterns in the data when observing a long time window. Take for example the measuring of a solar plant. On a daily basis it is obvious to see the sun rising and setting, depending on the voltage of the panels. Starting at 0 at night the voltage is rising before noon and descending in the afternoon. This is one representation in the data. But there could be more represenations hidden, which are not likely to see. The shadow of a tree wandering over the panels happening every day or a one time event like the snow covering the plant. \\
These variations in data are not always visible for a human and even less possible to label them accordingly. Like \cite{bengio_representation_2013} mentioned it is important for artificial intelligence to detect these representations in data by machines. A machine should be able to extract information hidden in the low-level sensor measurings and continue working with the representations instead of the raw data. This is according to the paper the main requirement for a good representation, to be able using it as an input to a supervised predictor.\\
Since the paper came out in 2013, several representation learning techniques were developed and some of them are directly applicable for time series data. In \cite{sun_survey_2021} the importance of machine learning in sensor data is emphasized. They sum up several deep learning techniques on data-driven soft-sensors. Soft-sensors represent hard to measure variables by adapting available sensor data. Their observation of industry processes is a rapidly changing field which demands data processing for a huge amount of data.
\subsection{Representation Learning Strategies}
The different RL strategies are listed, explained and compared.\\
\subsubsection{MLP}
Proposes a method called INRAD using a simple multi-layer perceptron to take time as input and output corresponding values, with representation error used as an anomaly score
\cite{jeong_time-series_2022}
\subsubsection{Contrastive Learning}
% Paper Debiased Contrastive Learning
Learning representations in time series data is tackled in a variety of ways. One solution according to \citeA{zhang_debiased_2024} is debiased contrastive learning. By comparing pairs of data points and rating the similarities as distances between the two, contrastive learning gets less dependant on labeled data. The data can be more general and the extracted representations are more robust. The pairs of data points are labeled as positive and negative pairs with a distance according to their similarities. With this distance they are put into a feature space where they form groups of data points. To minimize the bias between representations multigranularity augmented view generation and expert knowledge are used during training. \\\\
Contrastive Representation Learning is also used to tackle anomaly detection in time series data by \citeA{darban_carla_2024}. They use CL combined with synthetic anomaly injection. CL enables them to capture patterns in time series data and the framework shows good results on common real world datasets. Similar to the previous paper, dissimilar pairs, the anomalies, build distant data points and similar data points are close to each other. In order to train the model artificial anomalies are injected which build distant pairs. In the next stage the classification is done by  the proximity of the neighbours in the representation space. Additionally anchor points representing the nearest and furthest neighbour are given from each representation. Their methodology is called CARLA.\\\\
\subsubsection{Autoencoder}
\citeA{nivarthi_multi-task_2023} are the first to use a Unified Autoencoder (UAE) for time series data, namely the power forecast of wind and solar plants. They contribute to the challenge of predicting the possible outcome of renewable energy in a newly created plant, either wind or solar. To do so a UAE is combined with a Task Embedding Neural Network (TENN) They examine the usability divided in Single-Task, Multi-Task and Zero-Shot Learning. The method was first published in \citeA{nivarthi_unified_2022}. It is then extended by convolutional layers instead of the fully connected neural network layers (UCAE-TENN)
and also Long Short-Term Memory layers (ULAE-TENN).\\\\
To overcome the challenge of poorly available time series data sets \cite{ma_survey_2023}, the model family MOMENT tries to learn general patterns on a pile of time series data \cite{goswami_moment_2024}. The pile is a collection of different datasets which they assembled for their pretraining. According to the paper minimal finetuning is needed to perform well on time series tasks like anomaly detection. They published the model and made the usage easily accessible with its own python library. The constructed time serie pile consists of a widespread list of domains including Weather measurements, sensor values and power consumption datasets. They also included data not connected with the previous like the tongue and finger movement of humans. The different tasks which the model is evaluated on are forecasting (long and short horizon), classification, anomaly detection and imputation. Except for short-horizon forecasting all tasks are managed well. \\\\
Realising few-shot anomaly detection of images is done by \citeA{schwartz_maeday_2024}. The method MAEDAY can detect objects newly added to the frames. To achieve this a masked autoencoder is used who recreates the former image but without the anomaly. The difference between the initial and reconstructed images is calculated and the object then visible.
This method is useful for its ability to detect anomalies with very few examples, making it a powerful tool in scenarios where labeled data is rare. \citeA{schwartz_maeday_2024} demonstrate the effectiveness of MAEDAY in various applications, showcasing its potential for real-world anomaly detection tasks.\\\\
To detect anomalies in healthcare data a variational recurrent autoencoder is used by \citeA{pereira_learning_2019}. The focus is on electrocardiogram (ECG) datasets. Their method tackles the challenge of finding anomalies in unlabelled time series data. They created an unsupervised framework where the model learns to represent the data and detect anomalies without needing labeled examples.
The VRAE model works by learning to reconstruct the input sequences. During training, they add noise to the input data, and the model tries to reconstruct the original, uncorrupted data. This helps the model learn more robust representations of the data. To detect anomalies, they cluster these learned representations and use the Wasserstein distance to identify outliers. Their approach was tested on the ECG5000 dataset and showed that it could effectively detect unusual heartbeats, performing better than previous methods that required labeled data. \\\\
Another approach using VRAE involves creating synthetic anomalies to improve the detection process. In their method, they use a two-level hierarchical latent space representation. First, they distill feature descriptors of normal data points into more robust representations using autoencoders (AEs). These representations are then refined using a variational autoencoder (VAE) that creates a family of distributions. From these distributions, they select those that lie on the outskirts of the normal data as generators of synthetic anomalies.
By generating these synthetic anomalies, they train binary classifiers to distinguish between normal and abnormal data. Their hierarchical structure for feature distillation and fusion helps create robust representations, enabling effective anomaly detection without needing actual anomalies during training. Their method performs well on several benchmarks for anomaly detection \cite{rivera_time-series_2022}. \\\\
A Autoencoder is used by \cite{pranavan_contrastive_2022} \\\\% TODO

\subsubsection{GPT based}
test
\subsubsection{Shapelet Learning}
% Shapelets: Shapelets are short, representative subsequences extracted from time series data.
% Shapelet Learning: Extraction, Evaluation and Selection of shapelets. Medling 
\citeA{beggel_time_2019} address the problem of detecting anomalies in time series data using a novel unsupervised method based on shapelet learning. This approach is particularly useful in scenarios where labeling data is difficult and expensive.
Their method learns representative features that describe the shape of time series data from the normal class and simultaneously learns to accurately detect anomalies. The objective function encourages the learning of a feature representation in which normal time series lie within a compact hypersphere, while anomalous observations lie outside the decision boundary. This is achieved through a block-coordinate descent procedure.
The advantage of their approach is that it can efficiently detect anomalies in unseen test data without retraining the model, by reusing the learned feature representation. Experimental results on multiple benchmark datasets demonstrate the robustness and reliability of the method in detecting anomalous time series, outperforming competing methods when the training data contains anomalies \cite{beggel_time_2019}.\\\\
In contrast, Alshaer et al. (2022) propose a method combining matrix profiles with shapelet learning to handle streaming time series data. The matrix profile efficiently identifies potential anomalies in real-time, and shapelet learning characterizes these anomalies accurately. This approach is particularly suited for environments requiring immediate anomaly detection, such as finance, healthcare, and industrial monitoring \cite{alshaer_detecting_2020}.\\\\
While both methods utilize shapelet learning, Beggel et al. focus on static datasets and robust feature representation, whereas Alshaer et al. emphasize real-time detection in dynamic, streaming environments.
