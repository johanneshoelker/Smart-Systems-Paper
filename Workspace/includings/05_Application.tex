\section{Application for Zero Shot Anomaly Detection on Multi Variate Time Series Data}\label{application}
In this chapter the found articles are filtered using the exclusion criteria defined in chapter \ref{criteria}. By excluding methods that haven't been tested with multiple input variables we answer RQ2. Methods that are not tested in Zero-Shot Learning Scenarios are also excluded which covers RQ4.

More precisely we want to know by the defined filter process which of the proposed RL types are best suited for Zero Shot Anomaly Detection in multi-variate time series data. In this chapter a selection of appropriate methods for Time Series Data Anomaly Detection out of \ref{review} is extracted. 
% TODO put formulas for a time series and the anomalies. maybe some anomaly score too

 % table
 In order to achieve a successful implementation in the next chapter only models which are available and well documented are chosen for further examination. In \ref{tab_rl_methods} the links to public repositories are listed. They are found by citations in the paper.
 \begin{table}
   \caption{Representation learning methodologies and the classification by exclusion criteria. Single Letter abbreviations are introduced for Transformer (T) and Clustering (C). The check boxes show if the method is tested with multiple input variables (MV). Methods tested on zero shot learning (ZSL). Methods providing an open source implemention (OSA).}\label{tab_rl_methods}
   \begin{longtable}[]{@{}llllll@{}}
\toprule\noalign{}
Method Name & Author & Concept & MV & ZSL & OSA \\
\midrule\noalign{}
\endhead
\bottomrule\noalign{}
\endlastfoot
INRAD & Jeong et al. & MLP & & \xmark & \cmark \\
OmniAnomaly & Su et al. & RNN & & \xmark & \cmark \\
CNN based method & Kravchik et al. & CNN & & \xmark & \\
CNN based method & He et al. & CNN & & \xmark & \xmark \\
CL based method & Zhang et al. & CL & & \cmark & \xmark \\
CARLA & Darban et al. & CL & & \xmark & \\
CL-TAD & Ngu et al. & CL & & \cmark & \\
COCA & Wang et al. & CL & & \xmark & \\
CL based method & Lee et al. & CL & & & \\
CL based method & Chen et al. & CL & & \xmark & \\
TS2Vec & Yue et al. & CL & \cmark & \cmark & \cmark \\
TimeAutoAD & Jiao et al. & CL & & \xmark & \\
ContrastAD & Li et al. & CL & & \xmark & \\
Dcdetector & Yang et al. & CL & & \xmark & \\
TRL-CPC & Pranavan et al. & CL & & \xmark & \\
TiCTok & Kang et al. & CL & & \xmark & \\
MGCLAD & Qin et al. & CL & & \xmark & \\
TriAD & Sun & CL & & \xmark & \\
UCAE-TENN & Nivarthi et al. & AE & & & \\
MAEDAY & Schwartz et al. & AE & & & \\
VRAE based method & Pereira et al. & AE & \cmark & \cmark & \xmark \\
AE based method & Ramirez et al. & AE & & & \\
AE based method & Provotar et al. & AE & \cmark & \xmark & \xmark \\
FuSAGNet & Han et al. & AE, GNN & \cmark & & \\
MSTVAE & Pham et al. & AE & \cmark & & \\
deep AOC & Mou et al. & AE & & & \\
AE based method & Zhang et al. & AE & & \cmark & \\
RANSynCoders & Abdulaal et al. & AE & & & \cmark \\
VRQRAE & Kieu et al. & AE & & & \xmark \\
TCN-AE & Thill et al. & AE & & & \\
MEGA & Wang et al. & AE & \cmark & & \\
DAEMON & Chen et al. & AE & \cmark & \cmark & \xmark \\
TSMAE & Gao et al. & AE & & & \\
GRU-AE & Gong et al. & AE & & & \\
MSCVAE & Yokkampon et al. & AE & \cmark & \cmark & \\
MOMENT & Goswami et al. & T & & \cmark & \cmark \\
TranAD & Tuli et al. & T & & & \cmark \\
Transformer based method & Ye et al. & T & & & \\
AnomalyTransformer & Xu et al. & T & & & \cmark \\
DATN & Wu et al. & T & & & \\
LLM based method & Li et al. & T & & & \\
TiSAT & Doshi et al. & T & & & \\
TCF-Trans & Peng et al. & T & & & \\
DCT-GAN & Li et al. & T & & & \\
& Li et al. & C & & & \\
& Beggel et al. & SL & & & \\
& Alshaer et al. & SL & & & \\
& Aota et al. & & & & \\
& Niu et al. & LSTM & & & \\
\end{longtable}

 \end{table}