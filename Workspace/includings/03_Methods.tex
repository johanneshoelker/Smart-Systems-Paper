\section{Systematic Literature Review}\label{methods}
% Why is the literature research important
A literature review to contribute in the development of an anomaly detection tool is presented in this paper. It provides an overview on the latest trends in representation learning and extracts the possible solutions addressing the problem described in \ref{intro}. The review conforms to the methodology presented by \cite{kitchenham_systematic_2009}. First the research questions are formulated. Finally Inclusion and Exclusion Criteria are formulated in order to filter the found literature for the application. The search process and the websites used are listed.

% Limitations
Further analysis with a systematic quality assessment and data collection like in \cite{kitchenham_systematic_2009} are excluded.
\subsection{Research question}
The covered topic includes different areas of machine learning, all being further developed in recent years.
In order to break it down into separate concerns the following research questions are formulated:
% Research questions
\begin{itemize}
  \item RQ1: How can representations be learned using artificial intelligence?
  \item RQ2: Which representation learning (RL) types can be used for multivariate time series?
  \item RQ3: How to use RL for anomaly detection?
  \item RQ4: Are the methods useful for Zero Shot Learning Scenarios?
\end{itemize}
These question form a path for further chapters. RQ1 and RQ2 are explained in section \ref{theory}. Answering RQ3 involves a literature review in section \ref{review} which presents useful methods. RQ4 is answered in section \ref{application}. The research questions build a basis for the formulation of the following Criteria.
\subsection{Inclusion and Exclusion Criteria}\label{criteria}
This paper focuses on published methods for anomaly detection in Zero Shot Scenarios on MVTSD. In order to structure the search for and selection of relevant articles, the necessary guidelines are formulated below. Articles that are considered in more detail must meet the following inclusion criteria:
\begin{itemize}
% Inclusion
\item IC1: Methods using a representation learning concept
\item IC2: Methods handling time series data
\item IC3: Methods used for Anomaly Detection
\item IC4: Published in recent years (< 6 years)
\end{itemize}
The chosen articles are examined in more detail. They are described and explained in \ref{review}. Using the gained knowledge all described articles are filtered by the following exclusion criteria in \ref{application}.
\begin{itemize}
% Exclusion
\item EC1: Methods not tested on Zero-Shot Learning
\item EC2: Methods designed for univariate data
\item EC3: Multiple publications reporting the same methodologies
\item EC4: Methods with restricted availability
\end{itemize}
Using these exclusion criteria ensures to find methodologies that meet the desired use case described in the research questions.

EC1 excludes methods that are not tested in a Zero-Shot Learning scenario. The second exclusion criteria filter for methods handling multiple input variables only. EC3 avoids duplicated papers and EC4 ensures that the method is publicly available and does contain a description on how to implement and reproduce the outcomes.
\subsection{Search process}
A manual search of specific conference proceedings and journal papers was made. Considering the pace on which new developments emerge in the area of machine learning the help of research tools was needed. Specifically in the field of anomaly detection the publications are made in recent years. This makes it difficult to assure finding every relevant paper.

% Consensus
The main tool used to find papers was Consensus, which is an academic search engine. They use large language models (LLMs) and purpose-built search technology. The chatbot is based on ChatGPT 4.0 and should answer questions based on papers including their reference. For reassuring the existence of the papers conventional bibliographies are used.
