\section{Discussion and Future Work}\label{discussion}
The presented paper has a few limitations which are discussed in this chapter. Additionally ideas for further research are given.

% Zero-Shot Limitation
Some of the found methods are not tested in Zero-Shot Scenarios in the article they are presented. The authors didn't focus on Zero-Shot Learning and didn't use appropriate datasets for this use case.  This doesn't mean they can not perform well on ZSL. Further research in generating ZSL scenarios with benchmark datasets need to be done with these methods.

% Multivariate Limitations
In comparison the adaptability of univariate methods for multivariate data is restricted. If a method designed for time series data with a single input variable needs to be used for multiple input variables a redesign is necessary. This can be extensive or rather simple depending on the architecture. To know how much redesign is necessary every method has to be reviewed in further research. However, nearly all of the found methods are trained and tested with MVTSD.

% Interdependance between multiple variables
Some models like MOMENT are handling input variables seperately. The interconnection between the different channels is not considered directly. The influence of one variable on the other can provide informations that may be important for AD and the learned representations in general. A successful approach inclduing the correlations between multiple input variables can be found in \cite{pranavan_contrastive_2022}.

%Datasets
Another important evaluation point is the selection of datasets. The transferability between time series datasets is difficult due to the fact that the data between domains is huge \cite{ma_survey_2023}. Time series can have a similar shape across different domains but that does not have to be true in general. For a good model fitting to the specfic context, it should be trained on similar data. This way robust representations can be learned that hold information about the specific domain.

% timestamp of the anomaly in the mideel \cite !

% Selection Process
The model selection in this paper could have been more systematically. Several models were excluded that could possibly be adapted to the presented use case. A model selection process with generating synthetic anomalies simplifies the search for an appropriate dataset. Such a model selection process for zero shot AD is presented in \cite{fung_model_2024}.

% Exclusion criteria
Not only the methods available as open source should be implemented and tested. Methods without a code example provided by the authors can be implemented based on the architecture presented in the paper. THis needs to be done in future development.

% Limitations on implementation
Testing pretrained models that are not trained on the domain of inverter data is an interesting attempt, which could potentially have an advantageous outcome. This way domains that have no data available can be tested on anomalies. However in this case it is not ensured to find all anomalies successfully. For robust representations that contain information about the data, they need to be trained and finetuned for the specfifc use case in future works.

During the research on how to use the methods for entirely new datasets, several difficulties occured. The code provided by the authors was mainly provided for replicating the found results. Given scripts took the main benchmark datasets that were mentioned in the paper as an input. Preparing entirely new datasets in order to train and use the models was tricky because of the missing knowledge of their software architecture. Providing additional data preprocessing guidelines for generalization is necessary for further development. However, preprocessing SMA inverter data for some methods and a forward pass through the models was successful. This work can be used for further development.

Additionally the implemention in this paper is done without further analysis of the results. The correctness of the detection can be rated and compared in future works.
