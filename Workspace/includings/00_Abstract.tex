\begin{abstract}
  The detection of anomalies in time series data is subject of current research in machine learning. Several methods are published in recent years which try to detect anomalies in various ways. Mostly the patterns that represent time series are learned. This is particularly useful for a general knowledge about the data. The concepts used for this include CNNs, Contrastive Learning, Autoencoders and Transformers. This paper provides an overview of the recent developments in anomaly detection and presents methods that try to detect anomalies in time series data. The found methods are evaluated with regard to Zero-Shot Learning and a multivariate input.
  Methods that are publicly available are investigated further using a completely new dataset with known anomalies provided by SMA. A proof of work in a Zero-Shot scenario without training the models is presented. None of the methods found the correct time point of an anomaly. 

  \keywords{Representation Learning  \and Zero Shot Learning \and Anomaly Detection \and Multivariate Time Series.}
\end{abstract}
