\section{Method}\label{method}
% Why is the literature research important
A literature research to contribute in the development of an anomaly detection tool is presented in this paper. It provides an overview of the latest trends in representation learning and extracts the possible solutions addressing the problem described in \ref{intro}. The review conforms to the methodology presented by \citeA{kitchenham_systematic_2009}. First the research questions are formulated. The search process and the websites used are listed. Finally Inclusion and Exclusion Criteria are formulated in order to filter the found literature for the application.
\subsection{Research questions}
The paper focuses on the following research question:\\\\
% Research question
RQ1: Which types of representation learning are possible for Zero Shot Anomaly Detection in multi variate time series applications?\\\\
TODO maybe insert methods part like in \cite{su_large_2024}. Using paper inclusion and exclusion criteria.
\subsection{Search process}
TODO cite Consensus for finding the literature
\subsection{Inclusion and Exclusion Criteria}
This paper focuses on multi variate time series data. Anomaly detection in Zero Shot Scenarios in particular. To filter the found literature, the following Criteria are used later in \ref{application}. 

%TODO criteria tablet

\label{tab_criteria}
\begin{list}
% Inclusion
\item IC1: Is the method using a representation learning concept. The focus is on methods learning representations in data using machine learning concepts.\\
IC2: Is the method used for Anomaly Detection. Sometimes the proposed methods perform well on different use cases. If one of the use cases is Anomaly Detection, the paper is included in the review.\\
IC3:  
% Exclusion
EC1: Open Source Availability \\
EC2: tested on Zero-Shot Learning
\end{list}
