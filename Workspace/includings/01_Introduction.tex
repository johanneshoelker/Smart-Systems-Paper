\section{Introduction}\label{intro}
% Occurence of sensors and their data
Nowadays sensors can be found everywhere and they become more popular across multiple domains. Gyroscopes, cameras, compasses and accelerometers are integrated in smartphones. Physical machines are tracking their movement through vibration sensors, health care systems in hospitals visualize the heart beat of a patience and voltmeters measure the generated power in a solar plant. Everytime sensor values are collected, time series data (TSD) is produced.

% Collections of sensors produce multivariate TSD
In some scenarios the measurements of different sensors are combined. Physical machines sometimes track vibration and motor rotations, health care systems visualize the heart beat and solar plants measure voltage and current.
Collections of different sensors measuring at a common time window produce Multivariate Time Series Data (MVTSD).

% dependance on MV-TSD
Applications that produce MVTSD may evaluate the data and further decisions tha lead to actions depend on a correct analysis.
Normally the data is consistent and values change constantly in repetetive patterns. This is when the machine, the patience health or the solar plant is functioning like it is supposed to. But sometimes the values change unpredicted because of differing surroundings or other influences. This can lead to serious situations. Machine measurements detect a potential fault which may break the machine. When the patient's heart beat changes its pattern the health of the patient is seriously endangered. And a solar plant may detect a decline in the generated power which should further influence the power consumption for a better efficiency.

% Importance of Anomaly Detection
Recognising and reacting to these changes in MVTSD can therefore be very important% TODO\citeA.
. But these interruptions occur in different forms. They can be recognised as outliers or they are hidden and not obviously seen as anomalies. In some cases they form shapes which never occured before. This raises the demand for a tool to detect anomalies in time series data without any further knowledge of the anomaly.

% Difficulty of Anomaly Detection in multi veriate time series data
% TODO

% Structure of this paper
A systematic literature research concerning the topic is conducted and the best choices are implemented on a test data set. In \autoref{methods} the basic methodology used in this paper is described and the main research questions are formulated. First basic necessary terms are explained in \autoref{theory}. In \autoref{review} the found papers and their methodologies are presented and explained. The methodologies are compared and evaluated for usability in the context of Zero-Shot Anomaly Detection in \autoref{application}. The implementation of suitable techniques is provided in \autoref{implementation}. Finally the results are discussed and concluded in \autoref{discussion}
