\section{Introduction}\label{intro}
In industrial settings several applications rely on multi variate time series data. This could be sensor measurements or machine state values. The machine reacts to these values and works according to them. Normally the values are changing constantly in repetetive patterns. This is when the measured data or the machine is running uninterrupted like it is supposed to. Sometimes the values change unpredicted because of differing surroundings or other influences. This can lead to faults which may break the machine.\\\\
Recognising and reacting to these changes can be important% TODO\citeA.
. Considering the example of a machine fault, the early detection can possibly prevent further damages. But interruptions occur in different forms. In some cases they never occured like this before. This raises the demand for a tool to detect anomalies in time series data without any further knowledge of the anomaly.\\\\
This paper conducts a detailed literature research to contribute in the development of a anomaly dection tool. It provides an overview of the latest trends in representation learning and extracts the possible solutions addressing the described probelm. The paper focuses on the following research question:\\\\
What are the different types of representation learning possible for Zero Shot Anomaly Detection in time series applications?\\\\
A literature research concerning the topic is conducted and the best choices are implemented on a test data set. We begin in \ref{theory} by defining necessary terms and how they are used. In \ref{review} the found papers and their methodologies are presented and explained. The methodologies are compared and evaluated for usability in the context of Zero-Shot Anomaly Detection in \ref{application}. The implementation of suiting techniques is provided in \ref{implementation}. Finally the results are discussed and concluded in \ref{discussion}
