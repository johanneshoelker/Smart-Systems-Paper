\section{Introduction}\label{intro}
Several applications rely on multi variate time series data. This could be sensor measurements or machine state values. The machine reacts to these values and works according to them. In the normal case the data is changing constantly in repetetive patterns. This is when the measured data or the machine is running uninterrupted like it is supposed to. Sometimes the values can change unpredicted because of different reasons. Faults can happen which may break the machine or the circumstances in which the application is running changed unpredictedly.\\\\
Recognising and reacting to these anomalies can be very important % TODO\citeA.
. Considering the example of a machine fault, the early detection can prevent possibly prevent further damages. But interruptions are not always the same. They can occur in different shapes which in some cases never occured like this before. This raises the demand for a tool to detect anomalies in time series data without any further knowledge of the anomaly.\\\\
Finding a good solution to this problem requires detailed literature research. This paper is trying to provide answers to the problem by extracting possible solutions out of the literature. Therefore the paper focuses on the following research question:\\\\
What are the different types of representation learning possible for Zero Shot Anomaly Detection in time series applications?\\\\
A literature research concerning the topic is conducted and afterwards the best choices are implemented on a test data set. We begin in \ref{theory} by defining the most important phrases and how they are used inside the paper. In \ref{review} the literature is searched for any paper or book providing a RL technique. The found techniques are compared and evaluated for usability at Zero-Shot Anomaly Detection in \ref{application}. The implementation of the best suiting techniques is provided in \ref{implementation}. Finally the results are discussed and concluded in \ref{discussion}
