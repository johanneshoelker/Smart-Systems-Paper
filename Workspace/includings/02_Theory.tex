\section{Definitions and Conventions}\label{theory}
The basic expressions used in this paper are explained in the following chapter. First Representation Learning is defined and the different approaches to find representations in data are explained. A description on how to evaluate the found representations is given. Afterwards Zero Shot Learning as well as Anomaly Detection is described and explained.
\subsection{Representation Learning}
% What are representations
Representation Learning (RL) tries to detect meaningful interconnections in data relevant for further data analysis. These interconnections represent abstract information, so called background knowledge \cite{lavrac_representation_2021}.

% cognitive representations
In neural networks representations are learned in every layer. Considering hidden layers representations are hardly comprehensible for humans. They are stored in weights and biases and build so called neural representations. The next abstraction level are spatial representations which are represented as points or regions in conceptual spaces. Transforming the information from conceptual spaces into words forms symbolic representations. The three types build cognitive repesentations \cite{gardenfors_conceptual_2000}.

% Knowledge discovery process
To extract symbolic or spatial features which are more comprehensible for humans a knowledge discovery process with different methods of machine learning and data mining methods are used. RL techniques are divided into Propositionalization as symbolic representations and Embeddings as numeric representations \cite[p. 4]{lavrac_representation_2021}.

% RL techniques
RL occurs in several machine learning areas. Depending on the underlying concept different strategies to extract representations can be found. They work different in detecting patterns and store them in different ways \cite{bishop_pattern_2006}.

% description and main goal
In the book of \citeA[p. 525]{goodfellow_deep_2016} a general detailed description of representation learning is given. They summarize that representations should make the subsequent learning tasks easier. This implies that to find the best fitting representation and the underlying representation learning technique, we need to know the task it should perform afterwards.

\subsubsection{Concepts}
% MLP
The most straight-forward approach to detect representations are Multi Layer Perceptrons (MLP). A input vector is processed by artificial neurons passed through several layers of neurons, interconnected between each layer. The adjustment of the interconnections using weights and biases enables a learning process where labeled data is learned using backpropagation. The learns a training data set and is used to label unknown data afterwards \cite{nielsen_neural_2015}

% CNN
Convolutional Neural Networks (CNN) are a variation of MLP building subsets of the input vector. THis is mainly used in image processing.

%RNN
Recurrent neural networks (RNN) are another vairation of the tradition feed-forward MLP. Every neuron has a additional input containing the previous state. THis is useful for time series data.

% Contrastive learning
One solution to learn representations is contrastive learning. Pairs of data points are labeled as similar and dissimilar. These data points are put into a feature space where the distance between the two represents their similarity. With a contrastive loss function and a label of similarity between two points, the model is trained by putting the similar data points together and separating dissimilar points. Using this method groups of similar data points are formed.

% Autoencoders
Autoencoders are another important method in representation learning. An autoencoder is a type of neural network used to learn efficient codings of input data in an unsupervised manner. It consists of an encoder that compresses the input into a latent-space representation and a decoder that reconstructs the input from this representation. The goal is to minimize the difference between the input and the reconstructed output.

% SRNN TODO

% Transformers
Transformers, initially developed for natural language processing tasks, have become a powerful tool in representation learning. They use self-attention mechanisms to weigh the significance of each part of the input data differently, enabling the model to capture long-range dependencies \cite{vaswani_attention_2017}.

%Another advanced form of representation learning is Generative Adversarial Networks (GANs). GANs consist of two neural networks, a generator and a discriminator, that are trained simultaneously. The generator creates data samples, while the discriminator evaluates them against real data samples. Through this adversarial process, the generator learns to produce increasingly realistic data, leading to rich and high-quality representations useful for data augmentation and creative tasks.\\\\
% Summary RL
In summary, representation learning can be achieved using different techniques, each suitable for different types of data and tasks. From neural networks and autoencoders to transformers, these methods provide the tools necessary to transform raw data into meaningful representations that facilitate further analysis and learning.
\subsubsection{Evaluation}\label{theory:evaluation}
% \cite{lavric} 1.4.1
This chapter describes how to evaluate the performance of a RL approach.
\citeA{bengio_representation_2013} describe what makes a representation "good". They list the following factors:
\begin{itemize}
  \item Smoothness
  \item Multiple Explanatory Factors
  \item A hierarchical organization of explanatory factors
  \item Semi-supervised learning
  \item Shared factors across tasks
  \item Manifolds
  \item Natural clustering
  \item Temporal and spatial coherence
  \item Sparsity
  \item Simplicity of factor dependencies
\end{itemize}
"We want to find properties of the data but at the same time we don't want to loose information about the input" \cite[S. 525]{goodfellow_deep_2016}
\subsection{Zero Shot Learning}
%Zero-shot learning involves training a model on certain classes and then testing its ability to recognize new, unseen classes without any retraining. In the context of anomaly detection, this means the model should be able to detect types of anomalies it has not encountered during the training phase.
% Zero Shot derived from Transfer learning
Zero Shot Learning is an extreme form of transfer learning \cite[S. 536]{goodfellow_deep_2016}. While transfer learning is the concept of transferring the knowledge and weights gained at one task using them at solving another task, Zero-Shot Learning means there are no samples for the other task. The transformation of knowledge can help solving tasks where there are few or no samples available. The gained knowledge is normally stored as representations in data. Representations which are abstract enough to not see a specific item but information about items which can be applicated to groups of items. This also means that Zero-Shot Learning is only possible because additional information has been discovered during training \cite[S. 536]{goodfellow_deep_2016}.

% First zero shot detection
\citeA{palatucci_zero-shot_2009} were the first to implement a successful Zero-Shot Detection followed by \citeA{socher_zero-shot_2013} who used semantic word vector representations to classify words in groups and to sort new words with an accuracy of 90\% with a fully unsupervised model.
% Quote Definition for ZSL in \cite[S. 1532]{nivarthi_unified_2022}\\%TODO
\subsection{Anomaly Detection}
% definition of gruhl by cause
Several definitions of anomalies in data can be found in literature. In this paper the definition of \citeA[S. 54]{gruhl_novelty_2022} is used. It seperates anomaly and novelty detection as different tasks. Anomalies can be understood as outliers from the regular class. But these anomalies can vary in their cause. If there is a specific cause and the anomalies occur in its own cluster, they form a novelty. If instead the outliers randomly occur with no specific root cause, they are called noise. The cause for noise then is of a different kind and cannot be classified.

% definiton by shape
Instead of dividing anomalies by their cause the shape of anomalies can vary in several ways. In real measurement data any shape is possible and it is totally unpredictable \cite{schwartz_maeday_2024}. For training purposes anomaly injection is crucial. Then the anomalies are simulated as point anomalies or subsequence anomalies. Point anomalies occur once and can be global or contextual. Subsequence anomalies on the other hand change the values in a given time window or on long term. They can be divided in seasonal, shapelet and trend anomalies. Seasonal and shapelet change the values in a limited time window, trend anomalies are changing all following values \cite[p. 9]{darban_carla_2024}. %Seasonal in time direction, shapelet the amplitudes

% definiton for this paper
In this paper we want to focus on single time events, which are in any case anomalies. Potentially being caused by an unknown process, they cannot be classified \cite{gruhl_novelty_2022}. This defines our goal as an anomaly detection task.

% ## Theoretischer Rahmen
% Definitionen und Konzepte: Detaillierte Erklärungen zu Representation Learning, Zero Shot Learning und Anomaly Detection.
% Methoden und Techniken: Überblick über die verschiedenen Ansätze und Methoden im Representation Learning (z.B. Autoencoder, CNNs, Word Embeddings).
