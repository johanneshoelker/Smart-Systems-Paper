\section{Introduction}\label{intro}
Several applications rely on multi variate time series data. This could be sensor measurements or machine state values. In these cases the data is changing constantly in a repetetive manner for a long time. This is when the measured data or the machine is running uninterrupted like it is supposed to. But all of a sudden, measurements or values can change unpredicted because of different reasons. Recognising and reacting to these changes can be very important (TODO Source). But interruptions are not always the same. They can occur in different shapes which in some cases never occured like this before. This asks for a tool to detect anomalies in time series data.\\
Finding a good solution to this problem requires detailed literature research. This paper is trying to provide answers to the problem by extracting possible solutions out of the literature. Therefore the paper focuses on the following research question:\\
What are the different types of representation learning possible for Zero Shot Anomaly Detection in time series applications?\\
This is what we want to find out by conducting a literature research concerning the topic and afterwards implementing the best choices on a test data set. We begin in \ref{theory} by defining the most important phrases and how we use them. In \ref{review} the literature is searched for any paper or book providing a RL technique. The found techniques are compared and evaluated for usability at Zero-Shot Anomaly Detection in \ref{application}. The implementation of the best suiting techniques is provided in \ref{implementation}. Finally the results are discussed and concluded in \ref{discussion}
