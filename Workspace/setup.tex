% used class: memoir and in article form for headers etc.
\documentclass[runningheads]{llncs}

% Inhaltsverzeichnis:
% \usepackage{amssymb}
% \usepackage{hyperref} %for hyperlinks

% Font from LLNCS
\usepackage[T1]{fontenc}
% \renewcommand{\familydefault}{\sfdefault}

% wenn unbekannte Silbentrennung dann in nächste zeile und große abstände
\sloppy

%Inserting Images:
\usepackage{graphicx}
\graphicspath{ {./images/} {./schematics} {../Setup/Aufnahmen} {../Setup/Arbeitsbereich}}

%For Highlighting Commands
\usepackage{listings}
\lstset{basicstyle=\tiny}

%Bibliography: APA
\usepackage[nosectionbib]{apacite}
\bibliographystyle{apacite}

%Equations
\usepackage{amsmath}

%Including PDFs
\usepackage{pdfpages}

%Für plots
% \usepackage{filecontents} % um eine csv datei zu lesen
\usepackage{pgfplotstable}
\pgfplotsset{compat=1.17}
\usepackage{siunitx}
\usepackage{booktabs} % For \toprule, \midrule and \bottomrule
\usepgfplotslibrary{units} % Allows to enter the units nicely
% geometrische Formen: tikz
% \sisetup{
%   round-mode          = places, % Rounds numbers
%   % round-precision     = 2, % to 2 places/
% }
%Für Anführungszeichen
\usepackage[autostyle=true,german=quotes]{csquotes}

%Abkürzungsverzeichnis
\usepackage[acronym,nonumberlist,nopostdot]{glossaries}
\makenoidxglossaries
% \newacronym{hmd}{HMD}{Head Mounted Display}

\renewcommand{\acronymname}{Abkürzungsverzeichnis}
\glstoctrue


%Tables:
% \renewcommand{\arraystretch}{1.5} %Für mehr Abstand in Tabellen
