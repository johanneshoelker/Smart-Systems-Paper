\chapter{Review}
% TODO
In this chapter the found literature is put into context. Starting with classical literature about the fundamental findings followed by actual Trends in the Area of Representation Learning. Finally the different Representation Learning Strategies are listed and compared.
\section{History}
In this chapter the fundamental literature about the topic is going to be discussed. \\
Sensors and comparable applications produce time series data points which on a closer look may not make sense. They can vary in an unforeseen way and for a short time window they may be completely random. We have to step back and observe longer time periods which could be days or weeks. Or, for very dense measuring it is shorter but there are way more data points to handle.\\
Sometimes it is possible for a human to see some patterns in the data when observing a long time window. Take for example the measuring of a solar plant. On a daily basis it is obvious to see the sun rising and setting, depending on the voltage of the panels. Starting at 0 at night the voltage is rising before noon and descending in the afternoon. This is one representation in the data. But there could be more represenations hidden, which are not likely to see. The shadow of a tree wandering over the panels happening every day or a one time event like the snow covering the plant. \\
These variations in data are not always visible for a human and even less possible to label them accordingly. Like \cite{bengio_representation_2013} mentioned it is important for artificial intelligence to detect these representations in data by machines. A machine should be able to extract information hidden in the low-level sensor measurings and continue working with the representations instead of the raw data.
\section{Trends}
%         Aktuelle Entwicklungen: Überblick über die neuesten Forschungsergebnisse und Trends.
\section{Representation Learning Strategies}
The different RL strategies are explained and compared.
