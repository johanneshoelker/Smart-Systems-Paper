\chapter{Theorie}\label{theory}
\section{Representation Learning}
Representation Learning mainly tries to detect interconnections in data, which represent meanings which are relevant for further data analysis. There are several representation learning techniques to detect patterns and to store them in different ways.\cite{lavrac_representation_2021} 1.3
divides techniques into Propositionalization and Embeddings.\\
Propositionalization:
Embeddings:

\subsection{Evaluation}
% \cite{lavric} 1.4.1
describes how to evaluate the applied RL approach in\\
\cite{bengio_representation_2013} describes what makes a representation "good". They list the following factors:\\
\begin{itemize}
  \item Smoothness
  \item Sparsity
\end{itemize}
\section{Zero Shot Learning}
detailed description
\section{Anomaly Detection}
detailed description
% TODO
% ## Theoretischer Rahmen
% Definitionen und Konzepte: Detaillierte Erklärungen zu Representation Learning, Zero Shot Learning und Anomaly Detection.
% Methoden und Techniken: Überblick über die verschiedenen Ansätze und Methoden im Representation Learning (z.B. Autoencoder, CNNs, Word Embeddings).
