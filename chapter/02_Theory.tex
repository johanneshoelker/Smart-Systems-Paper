\chapter{Definitions and Conventions}\label{theory}
\section{Representation Learning}
Representation Learning mainly tries to detect interconnections in data, which represent meanings which are relevant for further data analysis. There are several representation learning techniques to detect patterns and to store them in different ways.\\
\cite{lavrac_representation_2021} 1.3
divides techniques into Propositionalization and Embeddings.\\
Propositionalization:
Embeddings:\\
Representations in data
In the book of \cite{goodfellow_deep_2016} this general detailed description of representation learning is given. They sum up that a representations should make the subsequent learning tasks easier. This implicates that to find a the best fitting representation and the underlying representation learning technique, we need to know the task it should perform afterwards.
\subsection{Evaluation}
% \cite{lavric} 1.4.1
describes how to evaluate the applied RL approach in\\
\cite{bengio_representation_2013} describes what makes a representation "good". They list the following factors:\\
\begin{itemize}
  \item Smoothness
  \item Sparsity
\end{itemize}We want to find properties of the data but at the same time we don't want to loose information about the input \cite{goodfellow_deep_2016} (S.525)
\section{Zero Shot Learning}
detailed description
\section{Anomaly Detection}
detailed description
% TODO
% ## Theoretischer Rahmen
% Definitionen und Konzepte: Detaillierte Erklärungen zu Representation Learning, Zero Shot Learning und Anomaly Detection.
% Methoden und Techniken: Überblick über die verschiedenen Ansätze und Methoden im Representation Learning (z.B. Autoencoder, CNNs, Word Embeddings).
