\chapter{Introduction}\label{intro}
Several applications rely on multi variate time series data. This could be sensor measurements or machine state values. In these cases the data is changing constantly in a repetetive manner for a long time. This is when the measured data or the machine is running uninterrupted like it is supposed to. But all of a sudden, measurements or values can change unpredicted because of different reasons. Recognising and reacting to these changes can be very important (Source). But interuptions are not always the same. They can occur in different shapes which in some cases never occured like this before. This asks for a tool to detect anomalies in time series data.\\
Finding a good solution to this problem requires detailed literature research. This paper is trying to provide answers to the problem by extracting possible solutions out of the literature. Therefore the paper focuses on the following research question:\\
What are the different types of representation learning possible for Zero Shot Anomaly Detection?\\

% ## Einleitung
% Hintergrund: Einführung in Representation Learning und Zero Shot Anomaly Detection.
% Ziel der Recherche: Klarstellung der Forschungsfrage und Zielsetzung der Literaturrecherche.
% **what are the different types of rl possible for zero shot anomaly detection**
% Focus on time series applications and sensor apps